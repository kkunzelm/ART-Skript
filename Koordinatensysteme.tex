\section{Koordinatensysteme}

Mit Hilfe von Koordinaten kann man geometrische Probleme algebraisch lösen, d. h. man kann mit geometrischen Objekten "`rechnen"'. 

Ein Koordinatensystem verbindet Punkte im Raum mit einer Menge von Zahlen. Es gibt zahlreiche Möglichkeiten, ein Koordinatensystem zu wählen. Beispiele sind das klassische karthesische Koordinatensystem, Polarkoordinaten, Zylinderkoordinaten oder sphärische Koordinaten. Die Fragestellung entscheidet, welches Koordinatensystem man verwendet. Man kann zwar prinzipiell mit allen Koordinatensystemen arbeiten, manche Problem lassen sich aber sehr elegant lösen, wenn man das passende Koordinatensystem auswählt. Möchte man z. B. die Oberfläche der Erde vermessen, bietet sich ein sphärisches Koordinatensystem an. Würde man alternativ ein karthesisches Koordinatensystem verwenden, wäre die Beschreibung der Erdoberfläche zwar ebenso möglich, die Form der Koordinaten wäre aber ungleich komplizierter.

\subsection{Vektoren}
\subsection{ Differentialform vom Grad 1, kurz 1-Form (engl. one-form)}
\subsection{Tensoren}


KHK todo:

Beschreiben von verschiedenen Koordinatensystemen wie z. B. bei Grinfeld ab Kapitel 3.6.

\subsection{Karthesische Koordinaten}

\subsection{Affine Koordinaten}

\subsection{Polarkoordinaten}

\subsection{Zylinderkoordinaten}

\subsection{Sphärische Koordinaten}

