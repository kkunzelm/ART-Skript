
\section{Einleitung}

Todo: Quelle ergänzen = Bertschinger, Introduction to Tensor Calculus for GR, MIT, 1999

Zum Verständnis der Allgemeinen Relativitätstheorie (ART) sind folgende Aspekte von besonderer Bedeutung. 

\begin{itemize}
	\item Raum-Zeit kann als vierdimensionaler Raum betrachtet werden. Durch den Einfluß der Gravitation ist das Raum-Zeit-Kontinuum "gekrümmt", d.h. im Gegensatz zur euklidschen Geometrie sind die Abstände zwischen den Koordinatenlinien nicht mehr "`äquidistant"'. Man spricht in diesem Zusammenhang davon, dass das Raum-Zeit-Kontinuum "`gekrümmt"' ist. Als Folge davon benötigt man zum Beschreiben physikalischer Zusammenhänge eine mathematische Form, die es erlaubt diese Gesetzmässigkeiten unabhängig von Koordinatensystemen zu beschreiben. Genau für diese Aufgabe eignen sich Tensoren sehr gut, so dass es erforderlich ist, die Tensornotation und Tensoralgebra zu lernen, wenn man sich mit ART beschäftigen möchte.
	
	\item Da man davon ausgeht, dass Raum-Zeit differenzierbar ist, was übrigens durch den Ausdruck Raum-Zeit-Kontinuum ausgedrückt wird, kann man für infinitesimal kleine Änderung der Raum-Zeit-Position lokal flache Koordinatensysteme formulieren, in denen dann näherungsweise wie in euklidschen, also flachen Räumen gerechnet werden kann.
\end{itemize}

Betrachtet man die gekrümmte Raum-Zeit der Allgemeinen Relativitätstheorie im Gegensatz zur speziellen Relativitätstheorie, bei der der dreidimensionale Raum nach euklidschen Kriterien beschrieben wird während die Zeit absolut ist, dann ändert sich die Interpretation der Bahnkurve von Partikeln wie folgt.

Nach der Newtonschen Interpretation wirkt die Gravitation auf Partikel im Raum und bewirkt eine Beschleunigung, die ihre Bahnkurve beeinflußt. Aus dem Gesichtspunkt der ART verursacht die Gravitation keine Beschleunigung der Partikel, sondern führt zu einer Krümmung von Raum-Zeit. Die Bahnkurve eines Partikels wird dann nicht unter dem Aspekt des Wirkens einer externen Kraft betrachtet. Solange keine zusätzlichen externen Kräfte wirken, bewegen sich die Partikel entlang der kürzesten Verbindung von zwei Punkten. Diese Verbindung wird als Geodäte bezeichnet. Aus dem Gesichtspunkt der SR mag diese Bahnkurve "gekrümmt" erscheinen. Im Bezugssystem der ART hingegen ist diese Bahnkurve eine Geodäte. 

Diese Zusammenhänge werden mit Hilfe der Einsteinschen Feldgleichungen beschrieben.


