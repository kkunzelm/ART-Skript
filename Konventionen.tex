\section{Konventionen und Vereinbarungen}

\begin{itemize}

\item SR = Spezielle Relativitätstheorie

\item ART = Allgemeine Relativitätstheorie

\item 
Zur Beschreibung eines Vektors kann man verschiedene Begriffe verwenden. Wir verwenden den Begriff "`Startpunkt"' für den Anfang, Ursprung oder Fußpunkt und den Begriff "`Endpunkt"' für die Spitze eines Vektors. 

\item 
Die Lichtgeschwindigkeit c kann zwar in den aus dem Alltag bekannten Einheiten $ [km/h] $ bzw. $ [km/s] $  angegeben werden. Dies führt jedoch zu unhandlichen Formeln. Aus diesem Grund wurde die Konvention eingeführt, die Lichtgeschwindigkeit $ [c = 1] $ zu setzen. 

\item
Lateinische Indizes, z. B. k, l oder m, werden verwendet, wenn drei Indexstufen \{0, 1, 2\} vorliegen.
Griechische Indizes, z. B. $ \mu $ oder $ \nu $ werden verwendet, wenn vier Indezstufen \{0, 1, 2, 3\} vorliegen.


\end{itemize}


